\documentclass[letterpaper,10pt,onecolumn]{article}
\usepackage[spanish]{babel}
\usepackage[utf8]{inputenc}
\usepackage{amsfonts}
\usepackage{amsthm}
\usepackage{amsmath}
\usepackage{mathrsfs}

\usepackage{enumitem}
\usepackage[pdftex]{color,graphicx}
\usepackage{hyperref}
\usepackage{listings}
\usepackage{calligra}
\usepackage{url}
%\usepackage{algpseudocode} 
\DeclareMathAlphabet{\mathcalligra}{T1}{calligra}{m}{n}
\DeclareFontShape{T1}{calligra}{m}{n}{<->s*[2.2]callig15}{}
\newcommand{\scripty}[1]{\ensuremath{\mathcalligra{#1}}}
\lstloadlanguages{[5.2]Mathematica}
\setlength{\oddsidemargin}{0cm}
\setlength{\textwidth}{490pt}
\setlength{\topmargin}{-40pt}
\addtolength{\hoffset}{-0.3cm}
\addtolength{\textheight}{4cm}

\begin{document}
\begin{center}

%\includegraphics[width=490pt]{header.png}\\[0.5cm]

\textsc{\LARGE M\'etodos Computacionales Avanzados}\\[0.1cm]

\large Jaime E. Forero Romero\\[0.5cm]

\end{center}

\large \noindent\textsc{Nombre del curso:}  M\'etodos Computacionales
Avanzados%Aqui  
                                %nombre del curso 
  
\noindent\textsc{C\'odigo del curso:} FISI 3028 / FISI 4028 %Aqui el codigo del
                                %curso 

\noindent\textsc{Unidad acad\'emica:} Departamento de F\'isica

\noindent\textsc{Periodo acad\'emico:} 201820 %Aqui el periodo,
                                %p.ej. 201510 

\noindent\textsc{Horario:} Mi 6:30 a 7:50 y Vi 6:30 a 7:50 %Aqui el horario, %p.ej. Ma y Ju, 10:00 a 11:20 

\noindent\rule{\textwidth}{1pt}\\[-0.3cm]

\normalsize \noindent\textsc{Nombre profesor(a) principal:} Jaime
E. Forero Romero%Aqui nombre del profesor principal 

\noindent\textsc{Correo electr\'onico:}
\href{mailto:je.forero@uniandes.edu.co}{\nolinkurl{je.forero@uniandes.edu.co}}
%Cambie address por su direccion de correo uniandes 

%\noindent\textsc{Horario y lugar de atenci\'on:} Ma y Ju 10:00 a
%11:00 AM, Oficina Ip208 %Aqui su horario y lugar de atencion, p.ej. Vi,
                     %15:00 a 17:00, Oficina Ip102  
%\\[-0.1cm]

\noindent\textsc{Nombre profesor(a) complementario(a):} %Aqui nombre
                                %del profesor complementario si aplica 

\noindent\textsc{Correo electr\'onico:}
\href{mailto:@uniandes.edu.co}{\nolinkurl{@uniandes.edu.co}}

%Cambie address por direccion de correo uniandes del profesor
%complementario 

%\noindent\textsc{Horario y lugar de atenci\'on:} %Aqui horario y
%lugar de atencion del profesor complementario, p.ej. Vi, 15:00 a
%17:00, Oficina Ip102 
%\\[-0.1cm]
%Repetir esto en caso de varios profesores complementarios

%\noindent\textsc{Nombre monitor(a):} %Aqui nombre del monitor si aplica

%\noindent\textsc{Correo electr\'onico:}
%\href{mailto:address@uniandes.edu.co}{\nolinkurl{address@uniandes.edu.co}}
%%Cambie address por direccion de correo uniandes del monitor 

%\noindent\textsc{Horario y lugar de atenci\'on:} %Aqui horario y
%lugar de atencion del monitor, p.ej. Vi, 15:00 a 17:00, Oficina Ip102 

\noindent\rule{\textwidth}{1pt}\\[-0.1cm]

\newcounter{mysection}
\addtocounter{mysection}{1}

\noindent\textbf{\large \Roman{mysection} \quad Introducci\'on}\\[-0.2cm]

%Este espacio es para hacer una introduccion al curso, evidenciando la
%propuesta metodologica. Debe ser clara y precisa. 

\noindent\normalsize Los m\'etodos computacionales fundamental el
trabajo en todas las \'areas t\'ecnicas y cient\'ificas,
ya sean principalmente experimentales o te\'oricas. Esto se debe en
gran parte a que la capacidad de utilizar computadoras de alto
rendimiento ha disminuido en costo monetario y en complejidad.

El curso de M\'etodos Computacionales Avanzados presenta estas
posibilidades computacionales a estudiantes de diferentes disciplinas
cient\'ificas. Para esto se porpone profundizar sus conocimientos
en dos \'areas: implementaci\'on de m\'etodos de aprendizaje estadístico (i.e. Machine Learning, algoritmos que aprenden de datos) y utilizaci\'on de t\'ecnicas de c\'omputo
masivamente paralelo.

Se asume que los estudiantes de este curso ya tienen conocimientos
b\'asicos en m\'etodos computacionales equivalentes al nivel del curso
M\'etodos Computacionales (FISI-2028). 
\\[0.1cm]

\stepcounter{mysection}
\noindent\textbf{\large \Roman{mysection} \quad Objetivos}\\[-0.2cm]

%En este espacio se debe precisar el ente visor del curso y el
%proposito ideal al finalizar el curso. 
\noindent\normalsize Los objetivos principales del curso son:

\begin{itemize}

  \item Demostrar aplicaciones de m\'etodos des estad\'istica bayesiana para el an\'alisis estad\'istico de datos. \\ [-0.6cm]
        \item Demostrar el uso de m\'etodos y librer\'ias de aprendizaje estad\'istico ( Machine
          Learning). \\[-0.6cm]
        \item Mostrar diferentes arquitecturas, paradigmas, lenguajes
          y librerias para c\'omputo masivamente paralelo.  \\[-0.6cm]
	\item Estudiar diferentes aplicaciones prácticas a problemas
          científicos y del contexto industrial de las metodologías
          computacionales modernas. \\[-0.6cm] 
\end{itemize}

\stepcounter{mysection}
\noindent\textbf{\large \Roman{mysection} \quad Competencias a
  desarrollar}\\[-0.2cm] 

%En este espacio se describen las habilidades que el estudiante desarrollara en el transcurso del curso.

\noindent\normalsize Al finalizar el curso, se espera que el
estudiante est\'e en capacidad de: 

\begin{itemize}
\item Manejar lenguajes modernos de computación numérica de
  bajo nivel (i.e. C/C++) y de alto nivel
  (i.e. Python/R/Julia). \\[-0.6cm]
\item Tener un esquema para pre-procesar, analizar y generar
  reportes científicos y técnicos a partir de diversas fuentes
  de datos utilizando métodos computacionales. \\[-0.6cm] 
\item Desplegar programas en paralelo en un cluster \\[-0.6cm].
\item Generar conocimiento a partir del modelamiento te\'orico
  y computacional de los conceptos vistos en clase.\\[-0.2cm]  
\end{itemize}

\stepcounter{mysection}
\noindent\textbf{\large \Roman{mysection} \quad Contenido por
  semanas}\\[-0.2cm] 

%Se expone de forma ordenada toda la tematica a tratar del curso. Debe planearse para 15 semanas.

\noindent\textbf{\textsc{Semana 1.}}
Binder. Jupyterlab. Fundamentos de python, numpy y matplotlib.
\\[-0.3cm]

\noindent\textbf{\textsc{Semana 2.}}
Fundamentos de probabilidad y Teorema de Bayes.\\
Referencia: Cap\'itulos 1 de DABT
\\[-0.3cm]

\noindent\textbf{\textsc{Semana 3.}} 
Estimaci\'on de par\'ametros con estad\'istica bayesiana. Algoritmo de Metropolis-Hastings.\\
Referencia: Cap\'itulo 29 de ITILA. Cap\'itulos 2 y 3 de DABT.  
\\[-0.3cm]  

\noindent\textbf{\textsc{Semana 4.}}
Algoritmos Monte Carlo eficientes: Nested Sampling, Hamiltoniano\\
Referencia: Cap\'itulo 30 de ITILA. Cap\'itulo 9 de DABT.
\\[-0.3cm] 

\noindent\textbf{\textsc{Semana 5.}} 
Selecci\'on de Modelos\\
Referencia: Cap\'itulo 4 de DABT.
\\[-0.3cm] 

\noindent\textbf{\textsc{Semana 6.}}  
Introducci\'on a Machine Learning. \\
Referencia: Cap\'itulos 1 y 2 de ISLR.
\\[-0.3cm] 

\noindent\textbf{\textsc{Semana 7.}}  
Regresi\'on lineal y cross-validation\\
Referencia: Cap\'itulos 3 y 5 de ISLR.
\\[-0.3cm]

\noindent\textbf{\textsc{Semana 8.}} 
Cross-validation y regularizaci\'on.\\
Referencia: Cap\'itulos 5 y 6 de ISLR.
\\[-0.3cm]  

\noindent\textbf{\textsc{Semana 9.}}
Semana de receso.
\\[-0.3cm] 

\noindent\textbf{\textsc{Semana 10.}} 
Clasificaci\'on.\\
Referencia: Cap\'itulo 4 ISLR. 
\\[-0.3cm]  

\noindent\textbf{\textsc{Semana 11.}} 
\'Arboles de Decisi\'on.\\
Referencia: Cap\'itulo 8 ISLR. 
\\[-0.3cm] 

\noindent\textbf{\textsc{Semana 12.}} 
Support Vector Machines.\\
Referencia: Cap\'itulo 9 ISLR.
\\[-0.3cm] 

\noindent\textbf{\textsc{Semana 13.}}
PCA, Clustering, K-means. t-SNE.\\
Referencia: Cap\'itulo 10 ISLR. 
\\[-0.3cm]

\noindent\textbf{\textsc{Semana 14.}} 
Fundamentos de C. Makefiles. Taxonom\'ia de arquitecturas para
c\'omputo en paralelo. M\'aquinas para c\'omputo en paralelo en
Uniandes.\\ 
Referencia: Cap\'itulo 11 del libro de ISTC. 
\\[-0.3cm]  


\noindent\textbf{\textsc{Semana 15.}} 
Fundamentos de programaci\'on en paralelo: MPI.\\
Referencia: Cap\'itulo 12 del libro de ISTC.
\\[-0.3cm]

\noindent\textbf{\textsc{Semana 16.}} 
Fundamentos de programaci\'on en paralelo: OPENMP.\\
Referencia: Cap\'itulo 13 del libro de ISTC.
\\[0.1cm]

\stepcounter{mysection}
\noindent\textbf{\large \Roman{mysection} \quad
  Metodolog\'ia}\\[-0.2cm] 

%Se describen las tecnicas y metodos para el desarrollo exitoso del curso.

\noindent\normalsize Cada semana tendremos una corta presentaci\'on
te\'orica (media hora aproximadamentes) para pasar a ejercitar esos
conceptos directamente en la computadora/cluster haciendo
ejercicios de pr\'actica (dos horas aproximadamente). 
Los estudiantes deben leer la bibliograf\'ia recomendada {\bf antes}
de las clases correspondientes.
\\[0.1cm]


\stepcounter{mysection}
\noindent\textbf{\large \Roman{mysection} \quad Criterios de
  evaluaci\'on}\\[-0.2cm] 

Las componentes que reciben calificaci\'on son las siguientes:

\begin{itemize}
\item 
Ejercicios para resolver y entregar en cada clase. 
El profesor eligir\'a cinco de estos ejercicios para ser calificados. 
Cada ejercicio cuenta un $10\%$ de la nota definitiva.

\item
Asistencia a clase. La asistencia se verifica a partir de la entrega de
los ejercicios hechos en clase. 
No entregar un ejercicio se asume como una falta a clase.
Cada falta recibe una nota de 0.0 y cada
asistencia recibe una nota de 5.0. 
El promedio de estas notas corresponde al $20\%$ de la nota
definitiva. 
Esta contribuci\'on a la nota definitiva ser\'a de cero (0.0) si se
dejaron de entregar {\bf ocho} o m\'as de estos ejercicios.
Es necesario asistir a clase para que la entrega se tome como
v\'alida. 
\item
Un examen final (con una componente escrita y otra de
programaci\'on) con un valor del $30\%$ de la nota definitiva.
\end{itemize}

Al comienzo del semestre se har\'a un examen ($5\%$ de bono
sobre la nota definitiva) para diagnosticar el conocimiento general
que ya tienen los estudiantes sobre los temas del curso. 
\\[0.1cm]



\newpage
\stepcounter{mysection}
\noindent\textbf{\large \Roman{mysection} \quad
  Bibliograf\'ia}\\[-0.2cm] 

%Indicar los libros y la documentacion guia.

\noindent\normalsize Bibliograf\'ia principal:

\begin{itemize}
\item [(DABT)] D.S. Sivia, J.Skilling, \textit{Data Analysis. A Bayesian Tutorial}, Second Edition, 2012, Oxford.\\[-0.6cm]
\item [(ITILA)] D. J. MacKay., \textit{Information Theory, Inference and
  Learning Algorithms}, 2003,
  Cambdrige. \\
  \url{http://www.inference.phy.cam.ac.uk/mackay/itila/}.
  \\[-0.6cm]  
\item [(ISLR)] G. James, D. Witten, T. Hastie, R. Tibshirani., \textit{An
  Introduction to Statistical Learning with Applications in R}, 2015,
  Springer. \\
  \url{http://www-bcf.usc.edu/~gareth/ISL/} \\[-0.6cm] 
\item [(ISTC)] F. T. Wilmore, E. Jankowski, C. Colina, \textit{Introduction
  to Scientific and Technical Computing}, 2017. CRC Press. (Biblioteca
  General - 502.85 I576)\\[-0.6cm] 

\end{itemize} 

\noindent\normalsize Bibliograf\'ia complementaria:

\begin{itemize}
\item I. Goodfellow, Y. Bengio, A. Courville., \textit{Deep Learning}, 
  2016, MIT.\\
  \url{http://www.deeplearningbook.org/}
  \\[-0.6cm] 
\item A. Tveito, H.P. Langtangen, B.F. Nielsen., \textit{Elements of
  Scientific Computing}, 2010.  (Biblioteca General, Recurso
  Electr\'onico 510. )\\[-0.6cm] 
\item R. L. Burden, J. D. Faires. \textit{Numerical analysis},
  2011. (Biblioteca General - 519.4 B862 2011)\\[-0.6cm]
\item O. Maimon and L. Rokach, \textit{The Data Mining and Knowledge
  Discovery Handbook}, 2010. (Biblioteca General, Recurso
  Electr\'onico 006.312)\\[-0.6cm]
\item M. Snir, \textit{MPI : the complete reference},
  1996. (Biblioteca General, 004.35 M637)\\[-0.6cm]
\item J. Sanders, E. Kandrot. \textit{CUDA by example: an
  introduction to general-purpose GPU programming}, 2010. (Biblioteca
  General - 005.275 S152)\\[-0.6cm]
\item D. Conway and J. M. White. \textit{Machine learning for
    hackers}, 2012.\\[-0.6cm]
\item Theano Development. \textit{Deep Learning Tutorial}
  \url{http://deeplearning.net/tutorial/}  \\[-0.6cm]
\item J. VanderPlas., \textit{Python Data Science Handbook}, 2016,
  O'Reilly.\\
  \url{https://github.com/jakevdp/PythonDataScienceHandbook} 
  \\[-0.2cm] 
\end{itemize}


\end{document}
