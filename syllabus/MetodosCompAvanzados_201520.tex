\documentclass[letterpaper,10pt,onecolumn]{article}
\usepackage[spanish]{babel}
\usepackage[utf8]{inputenc}
\usepackage{amsfonts}
\usepackage{amsthm}
\usepackage{amsmath}
\usepackage{mathrsfs}
\usepackage{empheq}
\usepackage{enumitem}
\usepackage[pdftex]{color,graphicx}
\usepackage{hyperref}
\usepackage{listings}
\usepackage{calligra}
\usepackage{url}
%\usepackage{algpseudocode} 
\DeclareMathAlphabet{\mathcalligra}{T1}{calligra}{m}{n}
\DeclareFontShape{T1}{calligra}{m}{n}{<->s*[2.2]callig15}{}
\newcommand{\scripty}[1]{\ensuremath{\mathcalligra{#1}}}
\lstloadlanguages{[5.2]Mathematica}
\setlength{\oddsidemargin}{0cm}
\setlength{\textwidth}{490pt}
\setlength{\topmargin}{-40pt}
\addtolength{\hoffset}{-0.3cm}
\addtolength{\textheight}{4cm}

\begin{document}
\begin{center}

\includegraphics[width=490pt]{header.png}\\[0.5cm]

\textsc{\LARGE M\'etodos Computacionales Avanzados}\\[0.1cm]

\large Jaime E. Forero Romero\\[0.5cm]

\end{center}

\large \noindent\textsc{Nombre del curso:}  M\'etodos Computacionales Avanzados%Aqui
                                %nombre del curso 
  
\noindent\textsc{C\'odigo del curso:} FISI XXXX %Aqui el codigo del
                                %curso 

\noindent\textsc{Unidad acad\'emica:} Departamento de F\'isica

\noindent\textsc{Periodo acad\'emico:} 201520 %Aqui el periodo,
                                %p.ej. 201510 

\noindent\textsc{Horario:} %Aqui el horario, %p.ej. Ma y Ju, 10:00 a 11:20 

\noindent\rule{\textwidth}{1pt}\\[-0.3cm]

\normalsize \noindent\textsc{Nombre profesor(a) principal:} Jaime
E. Forero Romero%Aqui nombre del profesor principal 

\noindent\textsc{Correo electr\'onico:}
\href{mailto:je.forero@uniandes.edu.co}{\nolinkurl{je.forero@uniandes.edu.co}}
%Cambie address por su direccion de correo uniandes 

%\noindent\textsc{Horario y lugar de atenci\'on:} Ma y Ju 10:00 a
%11:00 AM, Oficina Ip208 %Aqui su horario y lugar de atencion, p.ej. Vi,
                     %15:00 a 17:00, Oficina Ip102  
%\\[-0.1cm]

\noindent\textsc{Nombre profesor(a) complementario(a):} Sebastian Perez Saaibi %Aqui nombre
                                %del profesor complementario si aplica 

\noindent\textsc{Correo electr\'onico:}
\href{mailto:spsaaibi@uniandes.edu.co}{\nolinkurl{spsaaibi@uniandes.edu.co}}
%Cambie address por direccion de correo uniandes del profesor
%complementario 

%\noindent\textsc{Horario y lugar de atenci\'on:} %Aqui horario y
%lugar de atencion del profesor complementario, p.ej. Vi, 15:00 a
%17:00, Oficina Ip102 
%\\[-0.1cm]
%Repetir esto en caso de varios profesores complementarios

%\noindent\textsc{Nombre monitor(a):} %Aqui nombre del monitor si aplica

%\noindent\textsc{Correo electr\'onico:}
%\href{mailto:address@uniandes.edu.co}{\nolinkurl{address@uniandes.edu.co}}
%%Cambie address por direccion de correo uniandes del monitor 

%\noindent\textsc{Horario y lugar de atenci\'on:} %Aqui horario y
%lugar de atencion del monitor, p.ej. Vi, 15:00 a 17:00, Oficina Ip102 

\noindent\rule{\textwidth}{1pt}\\[-0.1cm]

\newcounter{mysection}
\addtocounter{mysection}{1}

\noindent\textbf{\large \Roman{mysection} \quad Introducci\'on}\\[-0.2cm]

%Este espacio es para hacer una introduccion al curso, evidenciando la
%propuesta metodologica. Debe ser clara y precisa. 

\noindent\normalsize Los m\'etodos computacionales fundamental el
trabajo en todas las \'areas t\'ecnicas y cient\'ificas,
ya sean principalmente experimentales o te\'oricas. Esto se debe en
gran parte a que la capacidad de utilizar computadoras de alto
rendimiento ha disminuido en costo monetario y en complejidad.

El curso de M\'etodos Computacionales Avanzados presenta estas
posibilidades computacionales a estudiantes de diferentes disciplinas
cient\'ificas. Para esto se porpone profundizar sus conocimientos
en tres \'areas: resoluci\'on de ecuaciones diferenciales,
implementaci\'on de m\'etodos de machine learning (i.e. algoritmos que
aprenden de datos) y utilizaci\'on de t\'ecnicas de c\'omputo
masivamente paralelo.

Se asume que los estudiantes de este curso ya tienen conocimientos
b\'asicos en m\'etodos computacionales equivalentes al nivel del curso
M\'etodos Computacionales (FISI-2028).
\\[0.1cm]

\stepcounter{mysection}
\noindent\textbf{\large \Roman{mysection} \quad Objetivos}\\[-0.2cm]

%En este espacio se debe precisar el ente visor del curso y el
%proposito ideal al finalizar el curso. 
\noindent\normalsize Los objetivos principales del curso son:

\begin{itemize}

        \item Presentar m\'etodos para la resoluci\'on de ecuaciones
          diferenciales parciales y ecuaciones diferenciales
          estoc\'asticas. \\[-0.6cm]
        \item Demostrar el uso de m\'etodos y librer\'ias de Machine
          Learning. \\[-0.6cm]
        \item Mostrar diferentes arquitecturas, paradigmas, lenguajes
          y librerias para c\'omputo masivamente paralelo.
	\item Estudiar diferentes aplicaciones prácticas a problemas
          científicos y del contexto industrial de las metodologías
          computacionales modernas. \\[-0.6cm] 
\end{itemize}

\stepcounter{mysection}
\noindent\textbf{\large \Roman{mysection} \quad Competencias a
  desarrollar}\\[-0.2cm] 

%En este espacio se describen las habilidades que el estudiante desarrollara en el transcurso del curso.

\noindent\normalsize Al finalizar el curso, se espera que el
estudiante est\'e en capacidad de: 

\begin{itemize}
\item Manejar lenguajes modernos de computación numérica de
  bajo nivel (i.e. C/C++) y de alto nivel
  (i.e. Python/R/Julia). \\[-0.6cm]
\item Tener un esquema para pre-procesar, analizar y generar
  reportes científicos y técnicos a partir de diversas fuentes
  de datos utilizando métodos computacionales. \\[-0.6cm] 
\item Desplegar programas en paralelo en un cluster, al
  igual que en sistemas de cómputo distribuídos tales como
  Amazon AWS.\\[-0.6cm] 
\item Generar conocimiento a partir del modelamiento te\'orico
  y computacional de los conceptos vistos en clase.\\[-0.2cm]  
\end{itemize}

\stepcounter{mysection}
\noindent\textbf{\large \Roman{mysection} \quad Contenido por
  semanas}\\[-0.2cm] 

%Se expone de forma ordenada toda la tematica a tratar del curso. Debe planearse para 15 semanas.


\noindent\normalsize\textbf{\textsc{Semana 1.}} Unix. Conceptos b\'asicos de
simulaci\'ones num\'ericas. Norma IEEE para aritm\'etica de punto
flotante. Repositorios. Makefiles.
\\[-0.3cm]   

\noindent\textbf{\textsc{Semana 2.}} 
Discretizaci\'on de ecuaciones
diferenciales parciales parab\'olicas (ecuaci\'on del calor) e
hiperb\'olicas (ecuaci\'on de onda).   
\\[-0.3cm]  

\noindent\textbf{\textsc{Semana 3.}} 
Discretizaci\'on de ecuaciones
diferenciales parciales el\'ipticas (ecuaci\'on de Laplace).   
\\[-0.3cm]  

\noindent\textbf{\textsc{Semana 4.}} 
Leyes de conservaci\'on como una expresi\'on hiperb\'olica. Problema
de Riemmann. M\'etodo de Godunov.
\\[-0.3cm]  

\noindent\textbf{\textsc{Semana 5.}}
Ecuaciones diferenciales estoc\'asticas.
\\[-0.3cm] 

\noindent\textbf{\textsc{Semana 6.}} 
M\'etodos Monte Carlo. Cadenas de Markov. 
\\[-0.3cm]  

\noindent\textbf{\textsc{Semana 7.}} 
Aprendizaje Supervisado: Árboles de Decisión, Clasificación, Ranking
Regresión, Redes Neuronales. 
\\[-0.3cm]  

\noindent\textbf{\textsc{Semana 8.}} 
Clustering: k-means, Jerárquico y Maximización del valor esperado (EM).
\\[-0.3cm] 

\noindent\textbf{\textsc{Semana 9.}} 
Reducción de Dimensionalidad y Predicción Estructurada: PCA, MDS, LDA;
Modelos Gráficos, Redes Complejas y Análisis de datos Topológico.  
\\[-0.3cm] 

\noindent\textbf{\textsc{Semana 10.}}  
Procesamiento Natural de Lenguaje (Naïve Bayes) y Aprendizaje Profundo
(Deep Learning). 
\\[-0.3cm] 

\noindent\textbf{\textsc{Semana 11.}}  
Fundamentos de programaci\'on en paralelo. Taxonomia de arquitecturas
para c\'omputo en paralelo. M\'aquinas para c\'omputo en paralelo en
Uniandes. 
\\[-0.3cm]  

\noindent\textbf{\textsc{Semana 12.}} 
C\'omo desplegar m\'aquinas de c\'omputo en La Nube. Ejemplos de
aplicaci\'on en Amazon Web Services y Docker. 
\\[-0.3cm]  

\noindent\textbf{\textsc{Semana 13.}} 
Fundamentos de programaci\'on en paralelo: MPI
\\[-0.3cm] 

\noindent\textbf{\textsc{Semana 14.}} 
Fundamentos de programaci\'on en paralelo: OPENMP
\\[-0.3cm] 

\noindent\textbf{\textsc{Semana 15.}} 
Fundamentos de programaci\'on en
paralelo: CUDA.
\\[-0.1cm]  


\stepcounter{mysection}
\noindent\textbf{\large \Roman{mysection} \quad
  Metodolog\'ia}\\[-0.2cm] 

%Se describen las tecnicas y metodos para el desarrollo exitoso del curso.

\noindent\normalsize Cada clase tendr\'a una corta presentaci\'on
te\'orica (30 minutos aproximadamente) para pasar a practicar todos
los conceptos directamente en la computadora/cluster a trav\'es de
ejercicios de pr\'actica (50 minutos aproximadamente). \\[0.1cm]


\stepcounter{mysection}
\noindent\textbf{\large \Roman{mysection} \quad Criterios de
  evaluaci\'on}\\[-0.2cm] 


En el curso tendr\'a cinco entregas de trabajos, cada una con un valor
del $20\%$ de la nota definitiva. Los temas de las entregas ser\'an
los siguientes:
\begin{enumerate}
\item Ecuaciones diferenciales parciales.
\\[-0.6cm]
\item Ecuaciones diferenciales estoc\'asticas.
\\[-0.6cm]
\item Machine Learning.
\\[-0.6cm]
\item C\'omputo en paralelo en MPI.
\\[-0.6cm]
\item Desarrollo de un proyecto propio que utilice c\'omputo
  masivamente paralelo y alg\'un otro de los temas vistos en clase. 
\\[-0.2cm]
\end{enumerate}



\stepcounter{mysection}
\noindent\textbf{\large \Roman{mysection} \quad
  Bibliograf\'ia}\\[-0.2cm] 

%Indicar los libros y la documentacion guia.

\noindent\normalsize Bibliograf\'ia principal:

\begin{itemize}
\item R. L. Burden, J. D. Faires. \textit{Numerical analysis},
  2011. (Biblioteca General - 519.4 B862 2011)\\[-0.6cm]
\item A. Tveito, H.P. Langtangem, B.F. Nielsen., \textit{Elements of
  Scientific Computing}, 2010.  (Biblioteca General, Recurso
  Electr\'onico 510. )\\[-0.6cm] 
\item O. Maimon and L. Rokach, \textit{The Data Mining and Knowledge
  Discovery Handbook}, 2010. (Biblioteca General, Recurso
  Electr\'onico 006.312)\\[-0.6cm]
\item M. Snir, \textit{MPI : the complete reference},
  1996. (Biblioteca General, 004.35 M637)\\[-0.6cm]
\item J. Sanders, E. Kandrot. \textit{CUDA by example: an
  introduction to general-purpose GPU programming}, 2010. (Biblioteca
  General - 005.275 S152)\\[-0.2cm]
\end{itemize} 

\noindent\normalsize Bibliograf\'ia complementaria:

\begin{itemize}
\item D. Conway and J. M. White. \textit{Machine learning for
    hackers}, 2012.\\[-0.6cm]
\item S.Bird. \textit{Natural Language Processing with
  Python}, 2009.\\[-0.6cm]
\item Theano Development. \textit{Deep Learning Tutorial}
  \url{http://deeplearning.net/tutorial/}  \\[-0.2cm]
\end{itemize}


\end{document}
