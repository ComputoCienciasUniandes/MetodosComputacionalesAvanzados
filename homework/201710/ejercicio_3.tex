%--------------------------------------------------------------------
%--------------------------------------------------------------------
% Formato para los talleres del curso de Métodos Computacionales
% Universidad de los Andes
%--------------------------------------------------------------------
%--------------------------------------------------------------------

\documentclass[11pt,letterpaper]{exam}
\usepackage{amsmath}
\usepackage[utf8]{inputenc}
\usepackage[spanish]{babel}
\usepackage{graphicx}
\usepackage{tabularx}
\usepackage[absolute]{textpos} % Para poner una imagen completa en la portada
\usepackage{hyperref}
\usepackage{float}

\newcommand{\base}[1]{\underline{\hspace{#1}}}
\boxedpoints
\pointname{ pt}

\extraheadheight{-0.15in}

\newcommand\upquote[1]{\textquotesingle#1\textquotesingle} % To fix straight quotes in verbatim



\begin{document}
\begin{center}
{\Large Universidad de los Andes - M\'etodos Computacionales Avanzados} \\
Ejercicio 3 - \textsc{Machine Learning}\\
5-05-2017\\
\end{center}



\vspace{0.3cm}


\noindent
La solución a este ejercicio debe subirse por SICUA antes de las 8:00PM
del viernes 5 de Mayo del 2017. 
Los c\'odigos deben encontrarse en un unico repositorio de \verb'github'
con el nombre \verb"NombreApellido_Ej3". Por ejemplo yo deber\'ia
crear un repositorio con el nombre
\verb"JaimeForero_Ej3". 

\noindent

En el repositorio debe estar un \'unico c\'odigo de python que
resuelve el problema propuesto. Debe utilizar la versi\'on \verb"0.18" de \verb"sklearn".

\vspace{0.3cm}

\begin{questions}
\question{\bf{Dorothea}}
\begin{itemize}
Utilizando los siguientes conjuntos de datos \url{http://archive.ics.uci.edu/ml/datasets/Dorothea}
\item (50 puntos) Construya un programa que utilice m\'etodos de Machine Learning para predecir si un compuesto qu\'imico es activo o inactivo, de tal manera que logre maximizar 
\item (50 puntos) Construya un programa que encuentre los atributos (que son representados por n\'umeros enteros) que m\'as influyen en hacer que el compuesto sea activo o inactivo.
\item (20 puntos) El bono, $B$, depende de la eficiencia de clasificaci\'on, $f$, del primer punto: $B = (f-0.5)\times 40$
\end{itemize}
\end{questions}

\end{document}

